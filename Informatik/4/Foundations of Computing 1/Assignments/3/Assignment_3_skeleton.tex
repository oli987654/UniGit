\documentclass{exercise}
\setcounter{exercise}{3}
\title{ \line(1,0){415} \\ Formale Grundlagen der Informatik I -\\ Assignment \theexercise\\
\line(1,0){415}}

\lefttitle{Universität Zürich\\Institut für Informatik \newline \\Student Name: \\Matrikel-Nr:}
\righttitle{FS 2020} 
\lowerright{}

\usepackage[DIV11]{typearea}
\usepackage[german]{babel}
\usepackage{graphicx}
\usepackage{a4wide}
\usepackage{amsmath}
\usepackage{amssymb}
\usepackage{amsthm}
\usepackage[scaled=.95]{helvet}
\usepackage{courier}
\usepackage{listings}
\usepackage{siunitx}
\usepackage{algorithm}
\usepackage{algorithmic}

\usepackage{tikz}
\usepackage{etoolbox}
\usepackage[utf8]{inputenc}

\newcommand{\answer}[1]{\vspace{0.25cm}\\\textbf{Answer:}~#1\vspace{0.25cm}}

\sloppy
\begin{document}
\maketitle

\begin{center}
	Hand out: 19.03.2020 - Due to: 09.04.2020\\\vspace{1em}
  Please upload your solutions to the Olat system.
\end{center}

\task{Binomial Coefficients}
  \subtaskTime{1} Without proof: What is the relation between Pascal's triangle and the binomial coefficients?
  \answer{

  }

  \subtaskTime{2} Please give a recursive formula to calculate $\binom{n}{k}$ with $n,k\in\mathbb{N}$ that appears reasonable. Please give a short explanation.
  \answer{

  }

\task{Mathematical Induction and Proofs}
  \subtaskTime{4} Please describe, how induction works as a proof. For what kind of problem is it well suited and for what kind of problem is it badly applicable and why?
  \answer{

  }

  \subtaskTime{2} Name the four steps that you have to perform in every induction.\\
  \answer{

  }
    
  \subtaskTime{9} Prove the following statements using induction:
    \begin{enumerate}
      \item $\sum\limits_{i=1}^n i = \frac{n(n+1)}{2}, \forall n \in \mathbb{N}^{+}$.
      \answer{

      }
      
      \item $\sum\limits_{i=1}^n i^2 = \frac{n(n+1)(2n+1)}{6}, \forall n \in \mathbb{N}^{+}$.
      \answer{

      }
      
    \end{enumerate}

  \subtaskTime{3} Now prove $\sum\limits_{i=1}^n i = \frac{n(n+1)}{2}, \forall n \in \mathbb{N}$ without induction.\\
  \answer{

  }

  \subtaskTime{2} Given $P(n) = (2^n < (n + 1)!)\;\forall n \in \mathbb{N}^{+}$. ($P$ takes a positive integer and returns a boolean.)
    \begin{enumerate}
      \item Write $P(2)$, is $P(2)$ true?
      \answer{

      }
      
      \item Write $P(k)$.
      \answer{

      }
      
      \item Write $P(k+1)$.
      \answer{

      }
      
      \item In a proof by mathematical induction that this inequality holds for all integers $n \geq 2$, what must be shown in the inductive step?
      \answer{

      }

    \end{enumerate}

\end{document}
