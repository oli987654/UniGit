\documentclass{exercise}

\setcounter{exercise}{5}
\newcommand{\topics}{Turing Machines, Undecidability}
\newcommand{\distdate}{23.11.2019}
\newcommand{\duedate}{08.12.2019}

\title{\line(1,0){415}\\
  Foundations of Computing II\\
  \Large Assignment \theexercise\\[1em]
  \large{\topics}\\
  \line(1,0){415}}

\lefttitle{Universit\"at Z\"urich\\Institut f\"ur Informatik\\[1em]
  \textsl{Student Name:}\\
  \textsl{Student Number:}}
\righttitle{Autumn 2020\\Sven Seuken\\Dennis Komm} 

\begin{document}

\maketitle

\begin{center}
  Distributed: \distdate\ -- Due Date: \duedate\\[1em]
  Upload your solutions to the OLAT system.\\[3em]
\end{center}

\task{Turing Machines}

Draw Turing machines (TMs) for the following languages and briefly explain how
your TMs work.

\subtask $L_1 = \{ a^kb^kw \mid k\in\nat^+\text{ and } w\in\{a,b\}^* \}$

  %% Insert your answer here

\subtask $L_2 = \{ w0w \mid w \in\{1,2\}^* \}$
  
  %% Insert your answer here

\nosubtask \textit{Hint:} It is sufficient to construct TMs with one tape each.
Recall that you can assume that the input word only contains letters from the
implicitly given alphabet; for instance, in a), there are only letters $a$ and
$b$ on the tape at the beginning.

\task{Turing Machines and Configurations}

Consider the following TM $M$ with $\text{Lang}(M)\subseteq\{0,1\}^*$.
\begin{center}
  \begin{tikzpicture}[x=3cm,y=2.2cm]
    \node[state]           at (0,0)  (2) {$q_2$};
    \node[state]           at (-1,0) (1) {$q_1$};
    \node[state,initial]   at (-2,0) (0) {$q_0$};
    \node[state]           at (1,0)  (3) {$q_3$};
    \node[state,accepting] at (2,0)  (4) {$q_4$};
    \path[->]
      (0) edge[loop above] node[label]
      {\begin{mlabels}
       \tmlabel{0}{0}{\rmove}
       \end{mlabels}} (0)
      (0) edge node[label]
      {\begin{mlabels}
       \tmlabel{1}{Y}{\lmove}
       \end{mlabels}} (1)
      (2) edge[loop above] node[label]
      {\begin{mlabels}
       \tmlabel{X}{X}{\rmove}\\
       \tmlabel{Y}{Y}{\rmove}
       \end{mlabels}} (2)
      (2) edge[bend left] node[label]
      {\begin{mlabels}
       \tmlabel{1}{Y}{\lmove}
       \end{mlabels}} (1)
      (1) edge[bend left] node[label]
      {\begin{mlabels}
       \tmlabel{0}{X}{\rmove}
       \end{mlabels}} (2)
      (1) edge[loop above] node[label]
      {\begin{mlabels}
       \tmlabel{X}{X}{\lmove}\\
       \tmlabel{Y}{Y}{\lmove}
       \end{mlabels}} (1)
      (2) edge node[label]
      {\begin{mlabels}
       \tmlabel{\blank}{Z}{\lmove}
       \end{mlabels}} (3)
      (3) edge[loop above] node[label]
      {\begin{mlabels}
       \tmlabel{X}{X}{\lmove}\\
       \tmlabel{Y}{Y}{\lmove}\\
       \tmlabel{Z}{Z}{\lmove}
       \end{mlabels}} (3)
      (3) edge node[label]
      {\begin{mlabels}
       \tmlabel{\blank}{Z}{\rmove}
       \end{mlabels}} (4);
    \end{tikzpicture}
\end{center}

\subtask Give the computation of $M$ (\ie, the unique sequence of configurations) on
  the following words; indicate when $M$ gets stuck or accepts.

  \begin{taskitems}
    \item $01$
    \item $001$
    \item $101$
    \item $00111$
  \end{taskitems}

  %% Insert your answer here

\subtask Describe $\text{Lang}(M)$ in words.

  %% Insert your answer here

\task{Diagonalization}

Consider the following two languages.
\begin{align*}
  L_{\text{diag},1} &= \{ w\in\{0,1\}^* \mid w = w_{2i} \text{ and } M_i \text{ does not accept } w_{2i} \}\;,\\
  L_{\text{diag},2} &= \{ w\in\{0,1\}^* \mid w = w_i \text{ and } M_{2i} \text{ does not accept } w_i \}\;.
\end{align*}
Here, we again assume that $M_i$ is the $i$th TM in a fixed ordering
and $w_i$ is the $i$th binary word in a fixed ordering over some given alphabet.
We see that both of these languages are constructed in a way that reminds us
of the language $L_{\text{diag}}$; the only part that is different is that we
do not speak about the main diagonal of the corresponding table, but two different
diagonals that are shallower or steeper, respectively.

\subtask For one of the two languages, prove that it is not recursively enumerable.

  %% Insert your answer here

\subtask For the other language, explain why the same argument as in a) is not
  valid to prove that this language is also not recursively enumerable.

  %% Insert your answer here

\task{More Diagonalization}

Let $L$ be some infinite language over $\{0,1\}$.  Explain how we can
identify a subset $L_{\text{diag},L}$ of $L$ such that $L_{\text{diag},L}$
is not recursively enumerable.

%% Insert your answer here

\task{Reductions}

$L_{\text{diag}}$ was the first language for which we showed that it is not recursively
enumerable and thus not recursive.  To prove that there are other
languages that are not recursively enumerable or not recursive, we use reductions.

In the lecture, we showed that $L_{\text{U}}$ is not recursive and argued as
follows.  We know that, if $L_{\text{U}}$ were recursive, then also its
complement $\overline{L}_{\text{U}}$ would be recursive.  Thus, if we succeed
in showing that $\overline{L}_{\text{U}}$ is not recursive, then $L_{\text{U}}$
cannot be recursive.  We then reduced $L_{\text{diag}}$ to
$\overline{L}_{\text{U}}$, that is, the problem of deciding whether a given
word is in $L_{\text{diag}}$ to deciding whether some word is in
$\overline{L}_{\text{U}}$.  If then we would have a TM $\overline{U}^*$ for deciding
$\overline{L}_{\text{U}}$ (that is, if this language were recursive), we could
use it to decide $L_{\text{diag}}$ (that is, this language would also be
recursive).

\nosubtask We want to slightly modify the original proof, but essentially prove the same statement.

\subtask Formally define the language $\overline{L}_{\text{diag}}$, that is, the complement of $L_{\text{diag}}$.

  %% Insert your answer here

\subtask Prove that $\overline{L}_{\text{diag}}$ is recursively enumerable.

  %% Insert your answer here

\subtask Reduce $\overline{L}_{\text{diag}}$ to $L_{\text{U}}$ to give an alternative proof that $L_{\text{U}}$ is not
  recursive.

  %% Insert your answer here

\task{More Reductions}

Consider the two languages
\begin{align*}
  L_3 &=\{(M,M',w) \mid M \text{ and } M' \text{ are TMs and } w\in \text{Lang}(M)\cap \text{Lang}(M')\}\;,\\
  L_4 &=\{(M,M',w) \mid M \text{ and } M' \text{ are TMs and } w\in \text{Lang}(M)\cup \text{Lang}(M')\}\;.
\end{align*}

\subtask Show that neither $L_3$ nor $L_4$ is recursive by giving a reduction from $L_{\text{U}}$.

  %% Insert your answer here

\subtask Give a reduction from $L_3$ to $L_{\text{U}}$.  Do so by using that the two TMs $M$ and $M'$
    can be simulated sequentially on the same word.

  %% Insert your answer here

\subtask Point out where we run into problems for a similar reduction as in exercise part
  b) from $L_4$ to $L_{\text{U}}$.  How can we deal with this problem?

  %% Insert your answer here

\task{Yet Another Reduction}

Consider TMs with exactly one accepting state and some fixed way to encode them.  Using a reduction,
show that the language
\begin{align*}
  L_5 = \{(M,w,i) \mid {}& M \text{ is a TM that visits its } i\text{th}\\
                         & \text{state at least once when processing } w \}\;.
\end{align*}
is not recursive.

%% Insert your answer here

\end{document}
