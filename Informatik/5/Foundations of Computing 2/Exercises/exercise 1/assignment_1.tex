\documentclass{exercise}

\setcounter{exercise}{1}
\newcommand{\topics}{Formal Languages, Automata, Regular Expressions}
\newcommand{\distdate}{21.09.2020}
\newcommand{\duedate}{04.10.2020}

\title{\line(1,0){415}\\
  Foundations of Computing II\\
  \Large Assignment \theexercise\\[1em]
  \large{\topics}\\
  \line(1,0){415}}

\lefttitle{Universit\"at Z\"urich\\Institut f\"ur Informatik\\[1em]
  \textsl{Student Name:}\\
  \textsl{Student Number:}}
\righttitle{Autumn 2020\\Sven Seuken\\Dennis Komm} 

\begin{document}

\maketitle

\begin{center}
  Distributed: \distdate\ -- Due Date: \duedate\\[1em]
  Upload your solutions to the OLAT system.\\[3em]
\end{center}

\task{Alphabets, Words, Languages}

\subtask Let $X = \{1321,2222,31\}$, $Y =\{\varepsilon,11,21\}$, and
  $Z = \{\varepsilon,u,bddd\}$ be languages over the alphabet $\{1,2,3,u,b,d\}$;
  let $\circ$ denote the concatenation operator.
  \begin{taskitems}
    \item Give the set of strings in $X^*$ that are of length $4$.
    \item Give the set of strings in $X \circ Y$ that are of length $6$.
    \item Give the set of strings in $(Y \cup Z) \circ X$ that are of length $5$ or less.
  \end{taskitems}

  %% Insert your answer here

\subtask We consider two languages $\{1\}$ and $\{2\}$ that contain only one word each.
  You are only asked to explain your arguments in words; no formal arguments are
  required. 
  \begin{taskitems}
    \item Explain why $(\{1\}^*\{2\}^*)^* =   (\{1,2\}^*)^2$.
    \item Explain why $(\{1\}^*\{2\}^*)^* \ne (\{1,2\}^2)^*$.
  \end{taskitems}

  %% Insert your answer here

\task{Finite Automata}

If you want to create graphs, \texttt{\textsc{Tikz}} is a nice tool to generate diagrams
from code.

\subtask Draw (either by hand or by using a drawing tool) a finite
  automaton (DFA) for each of the following languages.
  \begin{taskitems}
    \item The language $L_1=\{aab,aaab,b\}$ over the alphabet $\{a,b\}$.
    \item The language $L_2$ over the alphabet $\{a,b,c\}$ consisting of all words that start with
      $aba$ and contain at least one $c$.
    \item The language $L_3$ over the alphabet $\{0,1\}$ that consists of the words that are the binary
      representation of even numbers.  All representations (except $0$) should start with a $1$; \eg,
			the word $100$ is in $L_3$, but $11$, $0100$, and $101$ are not in $L_3$.
  \end{taskitems}

  %% Insert your answer here

\newpage

\subtask It is important to get the quantifiers straight in this context.  All the following
  languages are over the alphabet $\{0,1\}$.
  \begin{taskitems}
    \item Draw a DFA for the language $L_1=\{01\}$.
    \item Draw a DFA for the language $L_2=\{0011\}$.
    \item This can be generalized for arbitrary natural numbers.  For a given $k\in\nat$,
      sketch how an automaton for the language $L_k=\{0^k1^k\}$ would look like. 
    \item However, explain on an intuitive level, in two or three sentences, where the problem lies, if one
      would want to create a DFA for the language $L=\{0^k1^k\mid k\in\nat\}$.  Note that $L$ contains all words
      $\varepsilon,01,0011,000111,\dots$.  Later, we will even \emph{prove} that there
      cannot be a DFA for this language.
  \end{taskitems}

  %% Insert your answer here

\task{Cycles in Finite Automata}

Prove \emph{by contradiction} that every DFA contains a cycle.

%% Insert your answer here

\end{document}
