\documentclass{exercise}

\setcounter{exercise}{2}
\newcommand{\topics}{Deterministic and Non-deterministic Automata and Regular Languages}
\newcommand{\distdate}{05.10.2020}
\newcommand{\duedate}{25.10.2020}

\title{\line(1,0){415}\\
  Foundations of Computing II\\
  \Large Assignment \theexercise\\[1em]
  \large{\topics}\\
  \line(1,0){415}}

\lefttitle{Universit\"at Z\"urich\\Institut f\"ur Informatik\\[1em]
  \textsl{Student Name:}\\
  \textsl{Student Number:}}
\righttitle{Autumn 2020\\Sven Seuken\\Dennis Komm} 

\begin{document}

\maketitle

\begin{center}
  Distributed: \distdate\ -- Due Date: \duedate\\[1em]
  Upload your solutions to the OLAT system.\\[3em]
\end{center}

\task{Deterministic Finite Automata}

For a word $w$, we denote the number of occurrences of the letter $a$ in $w$ by $|w|_a$;
for instance, we have $|1010110|_0=3$ and $|xyzxzzy|_x=2$.  Draw DFAs for the following
languages.

\subtask $L_1=\{ w\in\{a,b\}^* \mid |w|_a + |w|_b \text { is even}\}$,

  %% Insert your answer here

\subtask $L_2=\{ w\in\{a,b\}^* \mid (|w|_a+2|w|_b)\bmod 4 = 3 \}$,

  %% Insert your answer here

\subtask $L_3=\{ w\in\{a,b,c\}^* \mid (|w|_a+2|w|_c)\bmod 5 = 0 \}$.

  %% Insert your answer here

\task{Size of Deterministic Finite Automata}

Consider the language
\[ L = \{ w\in\{a,b\}^* \mid w=xay \text{ with } x\in\{a,b\}^* \text{ and } y\in\{a,b\}^2\}, \]
\ie, the language that contains all words over the alphabet $\{a,b\}$ which
have at least three letters and an $a$ at the third-to-last position.  Prove
that any DFA for $L$ has to have at least four states in the following way.

Take a set of four words (that are not necessarily in $L$) and show that no two
of them are allowed to end in the same state of any DFA for $L$.  To this end, for
any two words $w_1$ and $w_2$, supply a \emph{shortest suffix} that implies that
one of the two words has to end in an accepting state while the other one must
not end in an accepting state. 

%% Insert your answer here

\task{Non-Deterministic Finite Automata}

\subtask Construct an NFA for the following languages. 

  \begin{taskitems}
    \item $L_1=\{w\in\{0,1,2\}^*\mid w \text{ contains the string } 111\}$,
    \item $L_2=\{w\in\{0,1,2\}^*\mid w \text{ ends with the string } 111\}$,
    \item $L_3$, which is matched by the regular expression $(1+0)^*1(0+1)11(0+1)(0+1)00$.

    \smallskip
    \textit{Hint:} Here, you do not have to use the construction from the lecture.
  \end{taskitems}

%% Insert your answer here

\subtask Consider the following NFA.

\begin{center}
  \begin{tikzpicture}[node distance=3cm]
    \node[initial,state]   (p)              {$p$};
    \node[state]           (q) [right of=p] {$q$};
    \node[accepting,state] (r) [right of=q] {$r$};
    \node[state]           (s) [right of=r] {$s$};
    \path[->] (p) edge [loop above]   node        {$1$}   (p)
                  edge                node        {$0$}   (q)
                  edge [bend right]   node[below] {$1$}   (r)
              (q) edge [loop above]   node        {$0$}   (q)
                  edge                node        {$0$}   (r)
              (r) edge [loop above]   node        {$0$}   (r)
                  edge [bend left=50] node        {$1$}   (s)
              (s) edge [loop above]   node        {$0,1$} (s)
              (s) edge [bend left=50] node        {$0$}   (r);
  \end{tikzpicture}
\end{center}

Use the powerset construction to transform this NFA into a DFA.

%% Insert your answer here

\task{Finite Automata and Regular Expressions}

\subtask Use the method that was introduced in the lecture to
  transform the following DFA into a regular expression.
  \begin{center}
    \begin{tikzpicture}[node distance=3cm]
      \node[initial,state]   (1)              {$1$};
      \node[accepting,state] (2) [right of=p] {$2$};
      \node[state]           (3) [right of=q] {$3$};
      \path[->] (1) edge               node        {$1$} (2)
                    edge[bend left=50] node        {$0$} (3)
                (2) edge[loop above]   node        {$0$} (2)
                    edge[bend right]   node[below] {$1$} (3)
                (3) edge[loop above]   node        {$0$} (3)
                    edge[bend right]   node[above] {$1$} (2);
    \end{tikzpicture}
  \end{center}
  Write down all steps and comment on what you are doing.

  %% Insert your answer here

\subtask Convert the regular expression $1+(0+1)^*+0$ into an NFA with
  $\varepsilon$-transitions.

  %% Insert your answer here

\subtask Convert the regular expression $01^*+1$ into an NFA with
  $\varepsilon$-transitions.

  %% Insert your answer here

\task{The Product Automaton}

Consider the two languages
\begin{align*}
  L_{10}  &= \{w \in \{0,1\}^* \mid w \text{ contains } 10\} \text{ and} \\
  L_{011} &= \{w \in \{0,1\}^* \mid w \text{ starts with } 011 \}.
\end{align*}
Construct the product automaton for $L_{10}\cap L_{011}$ with the technique
described in the lecture.

%% Insert your answer here

\task{Non-Regularity}

\subtask In the last assignment (more specifically, exercise 1.2(b)iv), we
  already sketched that there is a problem when designing a DFA for the
  language $L_{01}=\{0^k1^k \mid k \in \nat\}$.  Now use the pumping lemma
  to prove that this language is indeed not regular.

  %% Insert your answer here

\subtask Again, using the pumping lemma, prove that the language
  $L_{\text{sq}}=\{0^k \mid k \text{ is a square}\}$ is not regular.

  %% Insert your answer here

\end{document}
