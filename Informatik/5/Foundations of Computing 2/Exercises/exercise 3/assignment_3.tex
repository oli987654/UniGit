\documentclass{exercise}

\setcounter{exercise}{3}
\newcommand{\topics}{Context-Free Grammars and Languages}
\newcommand{\distdate}{26.10.2020}
\newcommand{\duedate}{01.11.2020}

\title{\line(1,0){415}\\
  Foundations of Computing II\\
  \Large Assignment \theexercise\\[1em]
  \large{\topics}\\
  \line(1,0){415}}

\lefttitle{Universit\"at Z\"urich\\Institut f\"ur Informatik\\[1em]
  \textsl{Student Name:}\\
  \textsl{Student Number:}}
\righttitle{Autumn 2020\\Sven Seuken\\Dennis Komm} 

\begin{document}

\maketitle

\begin{center}
  Distributed: \distdate\ -- Due Date: \duedate\\[1em]
  Upload your solutions to the OLAT system.\\[3em]
\end{center}

\task{Context-Free Grammars and Languages}

As defined in the lecture, for a word $w$, $w^{\rev}$ denotes the ``reversal''
of $w$; for instance, $(aabbba)^{\rev}$ is $abbbaa$.  Furthermore, as already defined
in Exercise 2.1, $|w|_a$ denotes the number of occurrences of the letter $a$ in
$w$.  Construct context-free grammars for the following languages.

\subtask $L_1 = \{w \in \{0,1\}^* \mid w\text{ ends with }01\text{ and }|w|_0\text{ is even}\}$

  %% Insert your answer here

\subtask $L_2 = \{w \in \{a,b\}^* \mid (2|w|_a+|w|_b) \bmod 5 = 0\}$

  %% Insert your answer here

\subtask $L_3 = \{w \in \{a,b\}^* \mid w \text{ ends with }aa\text{ or }w = w^{\rev} \}$

  %% Insert your answer here

\subtask As defined in the lecture, a grammar is called a ``regular grammar''
  if it has only productions of the form $X\to aY$, $X\to a$, $X\to\varepsilon$,
  where $X$ and $Y$ are non-terminals and $a$ is a terminal.  As the name
  suggests, exactly the regular languages allow for regular grammars.  For
  which of the above languages $L_1$, $L_2$, and $L_3$ can you give regular
  grammars?  How surprising is your result?

  %% Insert your answer here

\task{Parsing Strings}

The CFG $G_4=(\{S,A,B\},\{0,1\},P,S)$ with
\begin{align*}
  P_4 = \{ & S \to A1B, \\
           & A \to 0A \mid \varepsilon,\\
           & B \to 0B \mid 1B \mid \varepsilon\}
\end{align*}
generates the language $L_4$ which corresponds to the regular expression $0^*1(0+1)^*$.

\subtask Give both the leftmost and rightmost derivations of $1101$.

  %% Insert your answer here

\subtask Write down the parse tree of $001101$.

  %% Insert your answer here

\task{Normal Forms}

\subtask Use the method presented in the lecture to eliminate all $\varepsilon$-productions
 of the CFG $G_5=(\{S,A,B,C,D\},\{a,b,c\},P_5,S)$ with
 \[ P_5 = \{ S\to ABCD, A\to\varepsilon, A\to BB, B\to AA, A\to a, B\to b, C\to bc, D\to \varepsilon\}. \]

  %% Insert your answer here

\subtask Use the method presented in the lecture to eliminate all unit productions in
  the CFG $G_6=(\{S,A,B,C,D\},\{a,b,c,d\},P_6,S)$ with
  \[ P_6 =\{ S\to ABC, A\to B, B\to C, B\to b, B\to bB, C\to D, D\to d\}. \]

  %% Insert your answer here

\subtask Use the method presented in the lecture to eliminate all useless symbols in
  the CFG $G_7=(\{S,A,B,C,D,E\},\{a,b,c,d\},P_7,S)$ with
  \begin{align*}
     P_7 =\{ & S\to A, S\to AaB, S\to BbA, B\to bB, A\to aa, A\to Ab,\\
             & C\to cD, D\to c, D\to Ad, D\to EE, D\to dd\}.
  \end{align*}

  %% Insert your answer here

\subtask Convert the CFG $G_8=(\{S,A,B,C\},\{a,b\},P_8,S)$ with
  \begin{align*}
    P_8 = \{ & S\to aS, S\to Sb, S\to Aa, S\to bbB, A\to aBb, A\to ab,\\
             & B\to bCa, B\to ba, C\to b\}
  \end{align*}
  into Chomsky normal form.

  %% Insert your answer here

\task{CYK Algorithm}

Consider the CFG $G_9=(\{S,A,B,C\},\{a,b\},P_9,S)$ in Chomsky normal form with
\begin{align*}
  P_9 = \{ & S \to AB \mid BC,\\
           & A \to BA \mid a,\\
           & B \to CC \mid b,\\
           & C \to AB \mid a\}.
\end{align*}

Use the CYK algorithm to determine whether each of the following strings is in $L(G_9)$.
  
\subtask $ababa$
 
  %% Insert your answer here

\subtask $baaab$
  
  %% Insert your answer here

\subtask $aabab$

  %% Insert your answer here

\end{document}
